% This is a comment.
% the region directly below this comment, up till the command \begin{document} is known as the 'preamble'
% basic setup
\documentclass{article}
\usepackage[english]{babel}
\usepackage[utf8]{inputenc}

% for mathematics
\usepackage{amsmath}
\usepackage{amsthm}
% define theorems, lemmas, etc
\newtheorem{theorem}{Theorem}
\newtheorem{lemma}{Lemma}
\newtheorem{corollary}{Corollary}
\newtheorem{definition}{Definition}
\newtheorem{example}{Example}
\usepackage{amssymb}

% for adjusting margins
\usepackage{geometry}
\geometry{
	a4paper,
 	left=26mm,
 	right=20mm,
 	top=33mm,
 	bottom=38mm
}

% for introducing urls
\usepackage{url}

% for colored text
\usepackage{color}

% for creating lists
\usepackage{enumerate}

% for import graphics
\usepackage{graphicx}

% include algorithm package
\usepackage[]{algorithm2e, setspace}

% change font to times new roman
%\usepackage{times}

% add padding to in between paragraphs
\setlength{\parskip}{1em}

% eliminate indent at start of paragraph
\setlength\parindent{0pt}

% title details
\title{QF4102 Financial Modelling and Computation Assignment 3}
%\date{}
\author{G01 Wang Zexin, Chen Penghao}

%~~~~~~~~~~~~~~~~~~~~~~~~~~~~~~~~~~~~~~~~~~~~~~~~~~~~~~~~~~~~~~~~~~~~~~~~~~~~~~
\begin{document}

% insert title
\maketitle
% make a new page
\newpage

\section{Transformed Black-Scholes PDE model}
Consider the \textbf{transformed} Black-Scholes PDE model:
\begin{equation*}
  \begin{cases}
    \frac{\partial u}{\partial t} + \frac{\sigma ^ {2}}{2}\frac{\partial ^ {2} u}{\partial x^{2}} + (r - q - \frac{\sigma^{2}}{2})\frac{\partial u}{\partial x} -ru = 0, & x \in (-\infty, \infty), t \in [0, T) \\
    u(x, T) = \varphi(x), & 
  \end{cases}
\end{equation*}

\subsection{Derivation of fully implicit scheme}
Evaluate the partial derivatives at $(x_{n}^{i}, t_{n})$ where $t_{n} = n\Delta t, x_{n}^{i} = i\Delta x, n \in [0, \frac{T}{\Delta t}), i \in [-x_{max}, x_{max}]$
$$ \text{Use the forward time finite difference formulae : } \left. \frac{\partial u}{\partial t} \right| _{(x_{n}^{i}, t_{n})} = \frac{u_{n+1}^{i} - u_{n}^{i}}{\Delta t} + O(\Delta t)$$
$$ \text{Use the centred space finite difference formulae : } \left. \frac{\partial u}{\partial x} \right| _{(x_{n}^{i}, t_{n})} = \frac{u_{n}^{i+1} - u_{n}^{i-1}}{2\Delta x} + O[(\Delta x)^{2}]$$
$$ \text{Use the centred space finite difference formulae : } \left. \frac{\partial^{2} u}{\partial x^{2}} \right| _{(x_{n}^{i}, t_{n})} = \frac{u_{n}^{i+1} - 2u_{n}^{i} + u_{n}^{i-1}}{(\Delta x)^{2}} + O[(\Delta x)^{2}]$$

The finite difference equation is hence : 
$$ \frac{u_{n+1}^{i} - u_{n}^{i}}{\Delta t} + O(\Delta t) + \frac{\sigma ^ {2}}{2}\frac{u_{n}^{i+1} - u_{n}^{i-1}}{2\Delta x} + (r - q - \frac{\sigma^{2}}{2})\frac{u_{n}^{i+1} - u_{n}^{i-1}}{2\Delta x} + O[(\Delta x)^{2}] -ru_{n}^{i} = 0$$
$$ \frac{u_{n+1}^{i} - u_{n}^{i}}{\Delta t} + O(\Delta t) + O[(\Delta x)^{2}] = -\frac{\sigma ^ {2}}{2}\frac{u_{n}^{i+1} - u_{n}^{i-1}}{2\Delta x} - (r - q - \frac{\sigma^{2}}{2})\frac{u_{n}^{i+1} + u_{n}^{i-1}}{2\Delta x} + ru_{n}^{i}$$
$$ \frac{U_{n+1}^{i} - U_{n}^{i}}{\Delta t} = rU_{n}^{i} - \frac{\sigma ^ {2}}{2}\frac{U_{n}^{i+1} - U_{n}^{i-1}}{2\Delta x} - (r - q - \frac{\sigma^{2}}{2})\frac{U_{n}^{i+1} + U_{n}^{i-1}}{2\Delta x}$$
$$ U_{n+1}^{i} = U_{n}^{i} + \Delta t[rU_{n}^{i} - \frac{\sigma ^ {2}}{2}\frac{U_{n}^{i+1} - U_{n}^{i-1}}{2\Delta x} - (r - q - \frac{\sigma^{2}}{2})\frac{U_{n}^{i+1} + U_{n}^{i-1}}{2\Delta x}]$$
$$ U_{n+1}^{i} = U_{n}^{i} (1+ r \Delta t) - \frac{\Delta t}{2\Delta x}[\frac{\sigma ^ {2}}{2}(U_{n}^{i+1} - U_{n}^{i-1}) + (r - q - \frac{\sigma^{2}}{2})(U_{n}^{i+1} + U_{n}^{i-1})]$$
$$ U_{n+1}^{i} = U_{n}^{i} (1+ r \Delta t) - U_{n}^{i+1}\frac{\Delta t(r-q)}{2\Delta x} + U_{n}^{i-1}\frac{\Delta t(r - q - \sigma^{2})}{2\Delta x}, \forall -(I-1) \le i \le (I-1)$$
$$ U_{n+1}^{i} = aU_{n}^{i-1}+ bU_{n}^{i} + cU_{n}^{i+1} , \forall -(I-1) \le i \le (I-1)$$
for $a = \frac{\Delta t(r - q - \sigma^{2})}{2\Delta x}, b = 1+ r \Delta t, c = \frac{\Delta t(r-q)}{2\Delta x}$
With the values of $U_{n}^{I}$ and $U_{n}^{I}$ specified, we can express the FDE into matrix form.
\[
\begin{bmatrix}
    x_{11} & x_{12} & x_{13} & \dots  & x_{1n} \\
    x_{21} & x_{22} & x_{23} & \dots  & x_{2n} \\
    \vdots & \vdots & \vdots & \ddots & \vdots \\
    x_{d1} & x_{d2} & x_{d3} & \dots  & x_{dn}
\end{bmatrix}
\begin{bmatrix}
    U_{n}^{-I+1}\\
    U_{n}^{-I+2}\\
    U_{n}^{-I+3}\\
    \hdotsfor{1} \\
    U_{n}^{I+1}\\
    U_{n}^{I}
\end{bmatrix}
=
\begin{bmatrix}
    U_{n+1}^{-I+1}\\
    U_{n+1}^{-I+2}\\
    U_{n+1}^{-I+3}\\
    \hdotsfor{1} \\
    U_{n+1}^{I-1}\\
    U_{n+1}^{I}
\end{bmatrix}
+
\begin{bmatrix}
    U_{n}^{-I}\\
    0\\
    \hdotsfor{1} \\
    U_{n}^{I}
\end{bmatrix}
\]

\section{Valuation of digital call option}

\subsection{Algo}

\begin{algorithm}[H]
\setstretch{1.5}
	\KwData{$S_0$, $X$, $r$, $T$, $\sigma$, $N$, $\Delta s$}
	\KwResult{$c_{\text{EDS III}}$, Option Premium}
	$S_{\max} = 4X$, 
	$\Delta t = \dfrac{T}{N}$, 
	$I = round \left ( \dfrac{S_{\max}}{\Delta s} \right)$\;
	
	\For {$j = 0, 1, \dots , N$} {
		$V^{0}_{j} = (S_{\max} - X)e^{-r(N-j) \Delta t}$\;
		$V^{I}_{j} = 0$\;
	}
	
	\For {$i = 0, 1, \dots , I$} {
		$V^{i}_{N} = \max(X - i\Delta s, 0)$\;
		$a_i = \dfrac{1}{1 + r \Delta t} \left [\dfrac{1}{2} \sigma^2 i^2 \Delta t \right ]$\;
		$b_i = \dfrac{1}{1 + r \Delta t} \left (1 - \sigma^2 i^2 \Delta t - (r-q) i \Delta t \right )$\;
		$c_i = \dfrac{1}{1 + r \Delta t} \left [ \dfrac{1}{2} ( \sigma^2 i^2 + (r-q)i) \Delta t \right ]$\;
		\For {$n = N-1, N-2, \dots, 0$} {
			$V_n^i = a_i V^{i - 1}_{n+1} + b_i V^i_{n+1} + c_i V^{i+1}_{n+1}$ \;
		}
	}
	$i_0 = round \left (\dfrac{S_0}{\Delta s} \right )$\;
	$c_{\text{EDS III}} = V_0^{i_0}$\;
	
\end{algorithm}



\end{document}