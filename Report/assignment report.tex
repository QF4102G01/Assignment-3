% This is a comment.
% the region directly below this comment, up till the command \begin{document} is known as the 'preamble'
% basic setup
\documentclass{article}
\usepackage[english]{babel}
\usepackage[utf8]{inputenc}

% for mathematics
\usepackage{amsmath}
\usepackage{amsthm}
% define theorems, lemmas, etc
\newtheorem{theorem}{Theorem}
\newtheorem{lemma}{Lemma}
\newtheorem{corollary}{Corollary}
\newtheorem{definition}{Definition}
\newtheorem{example}{Example}
\usepackage{amssymb}

% for adjusting margins
\usepackage{geometry}
\geometry{
	a4paper,
 	left=26mm,
 	right=20mm,
 	top=33mm,
 	bottom=38mm
}

% for introducing urls
\usepackage{url}

% for colored text
\usepackage{color}

% for creating lists
\usepackage{enumerate}

% for import graphics
\usepackage{graphicx}

% include algorithm package
\usepackage[]{algorithm2e, setspace}

% change font to times new roman
%\usepackage{times}

% add padding to in between paragraphs
\setlength{\parskip}{1em}

% eliminate indent at start of paragraph
\setlength\parindent{0pt}

% title details
\title{QF4102 Financial Modelling and Computation Assignment 3}
%\date{}
\author{G01 Wang Zexin, Chen Penghao}

%~~~~~~~~~~~~~~~~~~~~~~~~~~~~~~~~~~~~~~~~~~~~~~~~~~~~~~~~~~~~~~~~~~~~~~~~~~~~~~
\begin{document}

% insert title
\maketitle
% make a new page
\newpage

\section{Transformed Black-Scholes PDE model}
Consider the \textbf{transformed} Black-Scholes PDE model:
\begin{equation*}
  \begin{cases}
    \frac{\partial u}{\partial t} + \frac{\sigma ^ {2}}{2}\frac{\partial ^ {2} u}{\partial x^{2}} + (r - q - \frac{\sigma^{2}}{2})\frac{\partial u}{\partial x} -ru = 0, & x \in (-\infty, \infty), t \in [0, T) \\
    u(x, T) = \varphi(x), & 
  \end{cases}
\end{equation*}

\subsection{Derivation of fully implicit scheme}
Evaluate partial derivatives at $(x_{n}^{i}, t_{n})$ where $t_{n} = n\Delta t, x_{n}^{i} = i\Delta x, n \in [0, \frac{T}{\Delta t}), i \in [-x_{max}, x_{max}], I_{max} = \frac{x_{max}}{\Delta x}$
$$ \text{Use the forward time finite difference formulae : } \left. \frac{\partial u}{\partial t} \right| _{(x_{n}^{i}, t_{n})} = \frac{u_{n+1}^{i} - u_{n}^{i}}{\Delta t} + O(\Delta t)$$
$$ \text{Use the centred space finite difference formulae : } \left. \frac{\partial u}{\partial x} \right| _{(x_{n}^{i}, t_{n})} = \frac{u_{n}^{i+1} - u_{n}^{i-1}}{2\Delta x} + O[(\Delta x)^{2}]$$
$$ \text{Use the centred space finite difference formulae : } \left. \frac{\partial^{2} u}{\partial x^{2}} \right| _{(x_{n}^{i}, t_{n})} = \frac{u_{n}^{i+1} - 2u_{n}^{i} + u_{n}^{i-1}}{(\Delta x)^{2}} + O[(\Delta x)^{2}]$$

The finite difference equation is hence : 
$$ \frac{u_{n+1}^{i} - u_{n}^{i}}{\Delta t} + O(\Delta t) + \frac{\sigma ^ {2}}{2}\frac{u_{n}^{i+1} -2u_{n}^{i} + u_{n}^{i-1}}{(\Delta x)^{2}} + (r - q - \frac{\sigma^{2}}{2})\frac{u_{n}^{i+1} - u_{n}^{i-1}}{2\Delta x} + O[(\Delta x)^{2}] -ru_{n}^{i} = 0$$
$$ \frac{u_{n+1}^{i} - u_{n}^{i}}{\Delta t} + O(\Delta t) + O[(\Delta x)^{2}] = -\frac{\sigma ^ {2}}{2}\frac{u_{n}^{i+1} -2u_{n}^{i} + u_{n}^{i-1}}{(\Delta x)^{2}} - (r - q - \frac{\sigma^{2}}{2})\frac{u_{n}^{i+1} - u_{n}^{i-1}}{2\Delta x} + ru_{n}^{i}$$
$$ \frac{U_{n+1}^{i} - U_{n}^{i}}{\Delta t} = rU_{n}^{i} - \frac{\sigma ^ {2}}{2}\frac{U_{n}^{i+1} -2U_{n}^{i} + U_{n}^{i-1}}{(\Delta x)^{2}} - (r - q - \frac{\sigma^{2}}{2})\frac{U_{n}^{i+1} - U_{n}^{i-1}}{2\Delta x}$$
$$ U_{n+1}^{i} = U_{n}^{i} + \Delta t[rU_{n}^{i} - \frac{\sigma ^ {2}}{2}\frac{U_{n}^{i+1} -2U_{n}^{i} + U_{n}^{i-1}}{(\Delta x)^{2}} - (r - q - \frac{\sigma^{2}}{2})\frac{U_{n}^{i+1} - U_{n}^{i-1}}{2\Delta x}]$$
$$ U_{n+1}^{i} = U_{n}^{i} (1+ r \Delta t) - \frac{\Delta t}{2(\Delta x)^{2}}[\sigma ^ {2}(U_{n}^{i+1} -2U_{n}^{i} + U_{n}^{i-1}) + \Delta x(r - q - \frac{\sigma^{2}}{2})(U_{n}^{i+1} - U_{n}^{i-1})]$$


$$ U_{n+1}^{i} = U_{n}^{i-1}[\frac{\Delta t(r - q - \frac{\sigma^{2}}{2})}{2\Delta x} -\frac{\sigma^{2}\Delta t}{2(\Delta x)^{2}}] + U_{n}^{i} [1+ r \Delta t + \frac{\sigma ^ {2} \Delta t}{(\Delta x)^{2}}] + U_{n}^{i+1}[- \frac{\Delta t(r - q - \frac{\sigma^{2}}{2})}{2\Delta x} -\frac{\sigma^{2}\Delta t}{2(\Delta x)^{2}} ]$$
$$ U_{n+1}^{i} = aU_{n}^{i-1}+ bU_{n}^{i} + cU_{n}^{i+1} , \forall I_{min}+1 \le i \le I_{max}-1$$
\hspace*{100pt} where $a = \gamma-\frac{\alpha}{2}, b = \beta + \alpha, c = -\gamma-\frac{\alpha}{2}, \alpha = \frac{\sigma^{2}\Delta t}{(\Delta x)^{2}}, \beta = 1+ r \Delta t, \gamma = \frac{\Delta t(r - q - \frac{\sigma^{2}}{2})}{2\Delta x}$\\[3mm]
The boundary conditions are as follows:\\
$$U_{n}^{I_{max}} = e^{-q(T-n \Delta t)}\exp(I_{max}\Delta x) - e^{-r(T-n \Delta t)}X \text{, when the underlying value is very large at} \exp(I_{max}\Delta x)$$
$$U_{n}^{I_{min}} = 0\text{, when the underlying value is very small at} \exp(I_{min}\Delta x)$$
\newpage
With the values of $U_{n}^{I_{min}}$ and $U_{n}^{I_{max}}$ specified, we can express the FDE into matrix form.
\[
\begin{bmatrix}
    b & c & \dots & \dots  & \dots & \dots & \dots\\
    a & b & c & \dots  & \dots & \dots & \dots\\
    \dots & a & b & c & \dots & \dots & \dots\\
    \vdots & \vdots & \vdots & \ddots & \vdots & \vdots & \vdots \\
    \dots & \dots & \dots & a & b & c & \dots\\
    \dots & \dots & \dots & \dots & a & b & c\\
    \dots & \dots & \dots & \dots & \dots & a & b\\
\end{bmatrix}
\begin{bmatrix}
    U_{n}^{I_{min}+1}\\
    U_{n}^{I_{min}+2}\\
    U_{n}^{I_{min}+3}\\
    \dots \\
    U_{n}^{I_{max}-3}\\
    U_{n}^{I_{max}-2}\\
    U_{n}^{I_{max}-1}
\end{bmatrix}
=
\begin{bmatrix}
    U_{n+1}^{I_{min}+1}\\
    U_{n+1}^{I_{min}+2}\\
    U_{n+1}^{I_{min}+3}\\
    \dots \\
    U_{n+1}^{I_{max}-3}\\
    U_{n+1}^{I_{max}-2}\\
    U_{n+1}^{I_{max}-1}
\end{bmatrix}
+
\begin{bmatrix}
    -aU_{n}^{I_{min}}\\
    0\\
    \vdots \\
    \vdots \\
    0\\
    -cU_{n}^{I_{max}}
\end{bmatrix}
\]

More concisely, we can name the tridiagonal matrix $A$ and the right hand side vector $F$ to express the FDE in this form: $AU_{n} = U_{n+1} + F \to U_{n} = A^{-1}(U_{n+1} + F)$

\subsection{Finite Difference Scheme Algorithm on fully implicit scheme}

\begin{algorithm}[H]
\setstretch{1.5}
	\KwData{$S_0$, $X$, $r$, $T$, $\sigma$, $I$, $N$, $x_{max}$}
	\KwResult{$c_{\text{IDS}}$, Option Premium}
	$\Delta t = \dfrac{T}{N}$, 
	$\Delta x = \dfrac{x_{max}}{I}$\;
	$\alpha = \frac{\Delta t(r - q - \frac{\sigma^{2}}{2})}{2\Delta x}$\;
	$\beta = 1+ r \Delta t$\;
	$\gamma = \frac{\sigma^{2}\Delta t}{2(\Delta x)^{2}}$\;
	$a = \alpha - \gamma$\;
	$b = \beta + \alpha$\;
	$c = -\alpha - \gamma$\;
	
	\For {$i = -I+1, -I+2, \dots, I-2, I-1 $} {
		$U_{N}^{i} = max(\exp(i\Delta x) - X, 0)$\;
	}
	
	Generate a tridiagonal matrix $A$ of dimension $(2I-1) * (2I-1)$, \\
	with $A_{i,i} = b\, \forall i = 1, 2, \dots 2I-1$,$A_{i,i-1} = b\, \forall i = 2, \dots 2I-1$, $A_{i,i+1} = b\, \forall i = 1, 2, \dots 2I-2$.\\
	
	\For {$j = N-1, N-2, \dots , 0$} {
		Generate a vector $F$ of length $(2I-1)$, with $F_{2I-1} = c\exp(-r(T-j \Delta t))(S_{max} - X)$, $F_{i} = 0$ otherwise\;
		$U_{j} = A^{-1} (U_{j+1} + F)$\;
	}
	$i_0 = round \left (\dfrac{\ln{S_0}}{\Delta x} \right )$\;
	$c_{\text{IDS}} = U_0^{i_0}$\;
	
\end{algorithm}

For the European vanilla call option with strike price \$5, time to maturity of 1 year, current underlier price of \$5.25, volatility of 30\%, risk free rate of 3\%, dividend yield of 10\%. Using a grid with values of x in the truncated domain $[-5, 5]$, with $N = 1500$ and $I$ taking values from $100$ to $1500$ with increments of 100, the option value estimates are obtained as tabulated below:
\begin{center}
	\begin{tabular}{| c | c | c |}
		\hline $I$ & $N$ & Option price\\
		[0.5ex]
		\hline 100 & 1500 & 0.522776022894615  \\
		\hline 200 & 1500 & 0.523228505305835  \\
		\hline 300 & 1500 & 0.523105617704949  \\
		\hline 400 & 1500 & 0.522656345235669  \\
		\hline 500 & 1500 & 0.522797471560081  \\
		\hline 600 & 1500 & 0.522898874252548  \\
		\hline 700 & 1500 & 0.522950382492270  \\
		\hline 800 & 1500 & 0.522963742476891  \\
		\hline 900 & 1500 & 0.522959159142429  \\
		\hline 1000 & 1500 & 0.522890228302100 \\
		\hline 1100 & 1500 & 0.522914330062009 \\
		\hline 1200 & 1500 & 0.522936922834744 \\
		\hline 1300 & 1500 & 0.522947345908465 \\
		\hline 1400 & 1500 & 0.522949937081634 \\
		\hline 1500 & 1500 & 0.522947446794060 \\
		\hline
	\end{tabular}
\end{center}

\begin{figure}[htbp!]
	\centering
	\includegraphics[scale=0.7]{smallPlot.jpg}
	\caption{European vanilla call option value estimates against $I$ with increments of 100}
\end{figure}

Since the value obtained at $I = 100$ is already quite close to the true value, it may be difficult to observe the convergence in the diagram of $I$ going from $100$ to $1500$ with increments of $100$ (figure 1). Hence, we plotted the diagram of $I$ going from $100$ to $1500$ with increments of $20$ in order to investigate further.

\begin{figure}[htbp!]
	\centering
	\includegraphics[scale=0.2]{largePlot.jpg}
	\caption{European vanilla call option value estimates against $I$ with increments of 20}
\end{figure}

From figure 2, it is obvious that the option value converges to the true value at around $\$0.52294$ as the fluctations become smaller and smaller as $I$ increases.

\subsection{American vanilla call option using PSOR}

For the American vanilla call option with same parameters as the above European vanilla call, we can use the projected SOR algorithm for the calculation of $U_{n}$ from $V_{n+1}$ with parameters $\epsilon = 1.0 * 10^{-6}, \omega = 1.3$.

\begin{center}
	\begin{tabular}{| c | c | c |}
		\hline $I$ & $N$ & Option price\\
		[0.5ex]
		\hline 100 & 1500 & 0.582849189993941 \\
		\hline 200 & 1500 & 0.583505358758593 \\
		\hline 300 & 1500 & 0.583457275858546 \\
		\hline 400 & 1500 & 0.582944640173874 \\
		\hline 500 & 1500 & 0.583166191075223 \\
		\hline 600 & 1500 & 0.583290840686203 \\
		\hline 700 & 1500 & 0.583345916640081 \\
		\hline 800 & 1500 & 0.583358092446549 \\
		\hline 900 & 1500 & 0.583351410333823 \\
		\hline 1000 & 1500 &0.583260621906101 \\
		\hline 1100 & 1500 &0.583286109957217 \\
		\hline 1200 & 1500 &0.583301049237271 \\
		\hline 1300 & 1500 &0.583302179614822 \\
		\hline 1400 & 1500 &0.583294042928569 \\
		\hline 1500 & 1500 &0.583280020736888 \\
		\hline
	\end{tabular}
\end{center}

We have plotted the values estimated with $I$ going from $100$ to $1500$ with increments of $100$ in figure 3.

\begin{figure}[htbp!]
	\centering
	\includegraphics[scale=0.2]{smallPlot2.jpg}
	\caption{American vanilla call option value estimates against $I$ with increments of 100}
\end{figure}

Similar to the previous section, we plotted another diagram of $I$ going from $100$ to $1500$ with increments of $25$ in order to investigate further as figure 4.

\begin{figure}[htbp!]
	\centering
	\includegraphics[scale=0.2]{largePlot2.jpg}
	\caption{American vanilla call option value estimates against $I$ with increments of 25}
\end{figure}

From figure 4, it is obvious that the option value converges to the true value at around $\$0.5834$ as the fluctations become smaller and smaller as $I$ increases.

\section{Valuation of digital call option}

\subsection{Pricing Algorithm of digital call option}

First, we implemented the function \texttt{BS\_DigitalCall(S0, X, r, q,  T, sigma)} for pricing the digital call option given:

\begin{algorithm}[H]
\setstretch{1.5}
	\KwData{$S_0$, $X$, $r$, $q$, $T$, $\sigma$}
	\KwResult{$c_{\text{DC}}$, Option Premium}
	$x = \dfrac{\log \left(\frac{S_0}{X}\right) + (r - q - \frac{\sigma^2}{2})T}{\sigma \sqrt{T}}$\;
	$c_{\text{DC}} = e^{-rT} N(x)$\;
\end{algorithm}

\subsection{Implementation of Monte Carlo Simulation without control variate}

Next up, we implemented the Monte Carlo simulation algorithm for estimating the 3-asset digital call option price, given the correlation matrix $C$ and the number of samples generated $N$, as well as the other parameters, $S_0$, $X$, $r$, $q$, $T$, same as those used in pricing of Digital Call option.

\begin{algorithm}[H]
\setstretch{1.5}
	\KwData{$S_0$, $X$, $r$, $q$, $T$, $\sigma$, $C$, $N$}
	\KwResult{$\texttt{MC\_noCV}$, Option Premium}
	\textit{First use the in-built function \emph{\texttt{randn}} in MATLAB to initiate a $3 \times N$ matrix \emph{$\mathbf{R}$.}} \\
	\textit{Each column of the matrix corresponds to the values of $x$ for the 3 assets in the portfolio.} \\
	$\mathbf{R} = \texttt{randn(3, N)}$\;
	\textit{Use Cholesky factorization to obtain a triangular matrix from the correlation matrix $C$} \\
	\textit{The in-build function \emph{\texttt{chol}} in MATLAB will lead to an upper triangular matrix.} \\
	$\mathbf{L} = \texttt{chol}(\mathbf{C})$ \;
	$\mathbf{E} =\mathbf{R}^T \mathbf{L} $ \;
	
	\textit{Construct the row matrix $\mathbf{p}$ for expected price.} \\
	\For {$j = 1,2,3$}{
		$\pmb{\mu}_j = r - \mathbf{q}_{j} - \frac{\pmb{\sigma}_{j}^2}{2}$ \;
		$\mathbf{p}_{j} = S_0 e^{\pmb{\mu}_jT}$ \;
	}
	\textit{Replicate the row matrices $\pmb{\sigma}$ and $\mathbf{p}$ into $N \times 3$ matrices $\mathbf{S}$ and $\mathbf{P}$} \\
	$\mathbf{S} = \texttt{repmat}(\pmb{\sigma}, N, 1)$ \;
	$\mathbf{P} = \texttt{repmat}(\mathbf{p}, N, 1)$ \;
	
	\textit{Thence, obtain the simulated terminal prices for the three assets, in an $N \times 3$ matrix $\mathbf{S}_T$.} \\
	\For {$i = 1 \dots N$}{
		\For {$j = 1,2,3$}{
			$w_{i,j} = \mathbf{S}_{i,j} \times \mathbf{E}_{i,j}$ \;
			$(\mathbf{S}_T)_{i,j} = \mathbf{p}_{i,j} \times\exp(w_{i,j}\sqrt{T})$
		}
	}
	
	\textit{Filter into a column matrix \emph{\texttt{max\_prices}} with only the maximum terminal prices in each sample.} \\
	$\texttt{max\_prices} = \texttt{max}(\mathbf{S}_T, \texttt{[]}, \texttt{2})$\;
	\textit{Filter out those in \emph{\texttt{max\_prices}} that are no more than strike $X$ into a column matrix \emph{\texttt{option\_values}}} \\
	$\texttt{option\_values} = (\texttt{max\_prices} > X)$ \;
	\textit{Finally, obtain the option value by discounting the average of the filtered terminal prices.} \\
	$\texttt{MC\_noCV} = \texttt{mean}(\texttt{option\_values}) \times e^{-rT}$ \;
	
\end{algorithm}

\subsection{Results from different number of price-path bundles and Comment}

Using the set of parameters, with the number of price-path bundles at 100, 1000, 10000, and 100000, and each round running 30 times of simulations, we obtain the following results in one of runs:

\begin{center}
	\begin{tabular}{| c | c | c | c |}
		\hline $X$ & $N$ & Option price estimate & Standard errors\\
		[0.5ex]
		\hline 8.5 & 100 & 0.88325 & 0.027956\\
		\hline 8.5 & 1000 & 0.87561 & 0.0095148\\
		\hline 8.5 & 10000 & 0.87745 & 0.0028249\\
		\hline 8.5 & 100000 & 0.87833 & 0.00081011\\
		\hline
		\hline 9.5 & 100 & 0.70923 & 0.053627\\
		\hline 9.5 & 1000 & 0.71379 & 0.014814\\
		\hline 9.5 & 10000 & 0.71402 & 0.0033345\\
		\hline 9.5 & 100000 & 0.71369 & 0.0013023\\
		\hline
		\hline 10.5 & 100 & 0.51114 & 0.058064\\
		\hline 10.5 & 1000 & 0.50082 & 0.013599\\
		\hline 10.5 & 10000 & 0.50434 & 0.0038309\\
		\hline 10.5 & 100000 & 0.50371 & 0.0013951\\
		\hline
	\end{tabular}
\end{center}

The option price estimate decreases for increase in value of $X$ for same values of $N$. This is explained by the pricing formula of the digital call option, where the value of $\dfrac{\partial c}{\partial X}$ is negative.

From the option price estimate for different strike prices, we do not observe an explicit overall increasing or decreasing trend within the same strike price. However, as $N$ gets larger, the change in estimated option price compared to the previous value of $N$ gets to decrease. 

This trend in the variance of the estimated values is more evidenced in the measurements of standard errors, which effectively decreased from smaller values of $N$ to larger values within the same strike price. Each time when $N$ increased by $10$ times, the value of standard errors is lowered to approximately between $\dfrac{1}{4}$ to $\dfrac{1}{3}$ of the previous value, which is similar to the value of $\dfrac{1}{\sqrt{10}}$.


\subsection{Variance Control and Effectiveness}
\subsubsection{Algorithm}
\begin{itemize}
	\item \textbf{Inputs:} $S_0$, $X$, $r$, $q$, $T$, $\sigma$, $C$, $N$
	\item \textbf{Output:} $\texttt{MC\_CV}$, Option Premium
	\item \textbf{Initiation: }\\ \vspace{1mm}
	$F_1 = \texttt{BS\_DigitalCall}(S_0^1, X, r, q_1, T, \sigma_1)$; \\ \vspace{1mm}
	$F_2 = \texttt{BS\_DigitalCall}(S_0^2, X, r, q_2, T, \sigma_2)$; \\ \vspace{1mm}
	$F_3 = \texttt{BS\_DigitalCall}(S_0^3, X, r, q_3, T, \sigma_3)$; \\ \vspace{1mm}
	$\bar{F} = \dfrac{F_1+F_2+F_3}{3}$
	
	\item Pilot simulation using the same Monte Carlo simulation method but for only $\dfrac{1}{5}$ of original sample size, to obtain $\beta$. The results of interest are:
	\begin{itemize}
	\item \texttt{pilotOptionValues}: the vector containing the simulated 3-asset digital call option values using samples in the pilot simulation; 
	\item \texttt{pilotBasketValues}: the vector of the mean digital call option value on each single asset in all samples in the pilot simulation.
	\end{itemize}
	\item Hence with the result, we can obtain $\beta$ by:
	\begin{itemize}
		\item First obtain the covariance matrix between the option values and the basket values by \\ \vspace{1mm} $\mathbf{C^{*}}$ = \texttt{cov}(\texttt{pilotOptionValues}, \texttt{pilotBasketValues})
		\item We would have:
	  \[\mathbf{C^{*}} =\begin{bmatrix}
		\sigma_{o}^2 & \sigma_{o,b} \\
		\sigma_{o,b} & \sigma_{b}^2
		\end{bmatrix} \]
		where $\sigma_{o}^2$ and $\sigma_{b}^2$ are the variance of the option values and the basket values respectively, and $\sigma_{o,b}$ is the covariance between option values and basket values.
		\item We hence could have an estimate for $\beta$ to use: $$\beta = \dfrac{\sigma_{o,b}}{\sigma_{b}^2}$$
	\end{itemize}
	\item Followed that, we carry out the actual computation for the final option value using the $\beta$ obtained.
	\item After obtaining the terminal stock price matrix $\mathbf{S}_T$, we also obtain the vectors for the 3-asset option values and the mean call option values for all samples.
	\begin{itemize}
		\item \texttt{optionValues}: the vector containing the simulated 3-asset digital call option values in the full simulation;
		\item \texttt{basketValues}: the vector containing the mean digital call option value on each single asset in all samples in the full simulation;
	\end{itemize}
	\item Then we have the vector \texttt{controlledOptionValues} obtained as $$\texttt{controlledOptionValues} = \texttt{optionValues} - \beta(\texttt{basketValues} - \bar{F})$$
	\item Taking the mean of the \texttt{controlledOptionValues} vector, we have the final option value $$\texttt{MC\_CV} = \texttt{mean}(\texttt{controlledOptionValues})$$
\end{itemize}

\subsubsection{Results}
Using the set of parameters and the control variate algorithm, with the number of price-path bundles at 100, 1000, 10000, and 100000, and each round running 30 times of simulations, we obtain the following results in one of runs:

\begin{center}
	\begin{tabular}{| c | c | c | c |}
		\hline $X$ & $N$ & Option price estimate & Standard errors\\
		[0.5ex]
		\hline 8.5 & 100 & 0.88306 & 0.026009\\
		\hline 8.5 & 1000 & 0.87657 & 0.0059658\\
		\hline 8.5 & 10000 & 0.87712 & 0.0015879\\
		\hline 8.5 & 100000 & 0.87799 & 0.00058616\\
		\hline
		\hline 9.5 & 100 & 0.71548 & 0.025147\\
		\hline 9.5 & 1000 & 0.71364 & 0.0061404\\
		\hline 9.5 & 10000 & 0.7145 & 0.0025493\\
		\hline 9.5 & 100000 & 0.71406 & 0.0005965\\
		\hline
		\hline 10.5 & 100 & 0.49626 & 0.028081\\
		\hline 10.5 & 1000 & 0.50429 & 0.0069935\\
		\hline 10.5 & 10000 & 0.50457 & 0.002156\\
		\hline 10.5 & 100000 & 0.50401 & 0.00039769\\
		\hline
	\end{tabular}
\end{center}

\subsubsection{Comments}
Using Monte Carlo simulation with control variate, we obtained results whose trends are to the previous runs with same set of parameters and without control variate. The similarity is in terms of:
\begin{itemize}
	\item Option premium converges with less variation as $N$ increases;
	\item The option premium decreases with increase in $X$; 
	\item The standard error of the results decreases for increase in $N$ within the same $X$. 
\end{itemize}

However compared to the runs with the same $X$ and same $N$ without control variate, we can see that the option premium estimates are of similar values, and the standard errors decreases. 

The following table summarizes the change in standard errors from methods with control variate and without control variate:

\begin{center}
	\begin{tabular}{| c | c | c | c | c |}
		\hline $X$ & $N$ & Standard errors & Standard errors & Percentage of original\\ 
		           & &WITHOUT control variate  & WITH control variate &\\
		[0.5ex]
		\hline 8.5 & 100 & 0.027956 & 0.026009 & 0.9304\\
		\hline 8.5 & 1000 & 0.0095148 & 0.0059658 & 0.6270\\
		\hline 8.5 & 10000 & 0.0028249 & 0.0015879 & 0.5621\\
		\hline 8.5 & 100000 & 0.00081011 & 0.00058616 & 0.7235\\
		\hline
		\hline 9.5 & 100 & 0.053627 & 0.025147 & 0.4689\\
		\hline 9.5 & 1000 & 0.014814 & 0.0061404 & 0.4144\\
		\hline 9.5 & 10000 & 0.0033345 & 0.0025493 & 0.7645\\
		\hline 9.5 & 100000 & 0.0013023 & 0.0005965 & 0.4580\\
		\hline
		\hline 10.5 & 100 & 0.058064 & 0.028081 & 0.4836\\
		\hline 10.5 & 1000 & 0.013599 & 0.0069935 & 0.5142\\
		\hline 10.5 & 10000 & 0.0038309 & 0.002156 & 0.5629\\
		\hline 10.5 & 100000 & 0.0013951 & 0.00039769 & 0.2850\\
		\hline
	\end{tabular}
\end{center}
We can see evident reduction in the standard errors of the results across all set of values of $X$ and $N$, when control variate is applied compared to when it is not.

From observation, the $\rho$, the estimated correlation coefficient between  the control variate and the option value for different strike prices are as follows:
\begin{center}
	\begin{tabular}{|c|c|c|}
	\hline $X$ & $\rho$ & $\sqrt{1-\rho^2}$\\
	\hline 8.5 & 0.68 & 0.7332 \\
	\hline 9.5 & 0.81 & 0.5864 \\
	\hline 10.5 & 0.88 & 0.4750 \\
	\hline
	\end{tabular}
\end{center}

According to the lecture notes, the theoretical value of the standard error after control variate is applied would be reduced to $\sqrt{1-\rho^2}$ of that of the original values obtained without using control variate. From the data of standard errors before and after applying control variate, we could see the ratio of after-control-variate values to the original values for the same strike price, are clustering around the theoretical percentage predicted using the values of $\sqrt{1-\rho^2}$

\end{document}